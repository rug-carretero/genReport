\documentclass{report}%
\usepackage[T1]{fontenc}%
\usepackage[utf8]{inputenc}%
\usepackage{lmodern}%
\usepackage{textcomp}%
\usepackage{lastpage}%
\usepackage{a4wide}%
\usepackage{listings}%
\usepackage{xcolor}%
\usepackage{courier}%
\usepackage{tabularx}%
\usepackage{hyperref}%
\usepackage{spverbatim}%
%
\UseRawInputEncoding%
\title{PHOENIX{-}6232}%
\author{symat}%
\date{2020{-}11{-}23}%
\lstset{tabsize = 4,showstringspaces = false,numbers = left,commentstyle = \color{darkgreen} \ttfamily,keywordstyle = \color{blue} \ttfamily,stringstyle = \color{red} \ttfamily,rulecolor = \color{black} \ttfamily,basicstyle = \footnotesize \ttfamily,frame = single,breaklines = true,literate = {\$}{{\textcolor{blue}{\$}}}1,numberstyle = \tiny}%
\definecolor{darkgreen}{rgb}{0,0.6,0}%
%
\begin{document}%
\normalsize%
\maketitle%
\tableofcontents%
\chapter{Root issue PHOENIX{-}6232}%
\label{chap:RootissuePHOENIX{-}6232}%
\section{Summary}%
\label{sec:Summary}%
Correlated subquery should not push to RegionServer as the probe side of the Hash join

%
\section{Description}%
\label{sec:Description}%
We were facing an interesting problem when a more complex query (with inner selects in the WHERE clause) succeeds alone, while the same query fails, if it is part of a join. I created a test table / query to reproduce the problem:\newline%
\begin{lstlisting}[language=sql]
DROP TABLE IF EXISTS test;
CREATE TABLE test (
      id INTEGER NOT NULL,
      test_id INTEGER,
      lastchanged TIMESTAMP,
      CONSTRAINT my_pk PRIMARY KEY (id));

UPSERT INTO test VALUES(0, 100, '2000-01-01 00:00:00.0');
UPSERT INTO test VALUES(1, 101, '2000-01-01 00:00:00.0');
UPSERT INTO test VALUES(2, 100, '2011-11-11 11:11:11.0');
\end{lstlisting} \ \newline%
\newline%
*Query 1:* Example query, running fine in itself:\newline%
\begin{lstlisting}[language=sql]
SELECT id, test_id, lastchanged FROM test T
WHERE lastchanged = ( SELECT max(lastchanged) FROM test WHERE test_id = T.test_id )

Returns:
+----+---------+-----------------------+
| ID | TEST_ID |      LASTCHANGED      |
+----+---------+-----------------------+
| 1  | 101     | 2000-01-01 01:00:00.0 |
| 2  | 100     | 2011-11-11 12:11:11.0 |
+----+---------+-----------------------+
\end{lstlisting} \ \newline%
\newline%
*Query 2:* Same query fails on the current master branch, when it is part of a larger (implicit) join:\newline%
\begin{lstlisting}[language=sql]
SELECT AAA.*
FROM(
  SELECT id, test_id, lastchanged FROM test T
  WHERE lastchanged = ( SELECT max(lastchanged) FROM test WHERE test_id = T.test_id )
) as AAA,
(
  SELECT id FROM test
) as BBB
WHERE AAA.id = BBB.id;


java.lang.IllegalArgumentException
	at org.apache.phoenix.thirdparty.com.google.common.base.Preconditions.checkArgument(Preconditions.java:128)
	at 
org.apache.phoenix.compile.TupleProjectionCompiler.createProjectedTable(TupleProjectionCompiler.java:66)
	at org.apache.phoenix.compile.QueryCompiler.compileSingleFlatQuery(QueryCompiler.java:663)
	at org.apache.phoenix.compile.QueryCompiler.compileJoinQuery(QueryCompiler.java:404)
	at org.apache.phoenix.compile.QueryCompiler.compileJoinQuery(QueryCompiler.java:302)
	at org.apache.phoenix.compile.
QueryCompiler.compileSelect(QueryCompiler.java:249)
	at org.apache.phoenix.compile.QueryCompiler.compile(QueryCompiler.java:176)
	at org.apache.phoenix.jdbc.PhoenixStatement$ExecutableSelectStatement.compilePlan(PhoenixStatement.java:504)
	at org.apache.phoenix.jdbc.PhoenixStatement$ExecutableSelectStatement.compilePlan(PhoenixStatement.java:467)
	at org.apache.phoenix.jdbc.PhoenixStatement$1.call
(PhoenixStatement.java:309)
	at org.apache.phoenix.jdbc.PhoenixStatement$1.call(PhoenixStatement.java:298)
	at org.apache.phoenix.call.CallRunner.run(CallRunner.java:53)
	at org.apache.phoenix.jdbc.PhoenixStatement.executeQuery(PhoenixStatement.java:297)
	at org.apache.phoenix.jdbc.PhoenixStatement.executeQuery(PhoenixStatement.java:290)
	at org.apache.phoenix.jdbc.PhoenixStatement.execute(Phoenix
Statement.java:1933)
	at sqlline.Commands.executeSingleQuery(Commands.java:1054)
	at sqlline.Commands.execute(Commands.java:1003)
	at sqlline.Commands.sql(Commands.java:967)
	at sqlline.SqlLine.dispatch(SqlLine.java:734)
	at sqlline.SqlLine.begin(SqlLine.java:541)
	at sqlline.SqlLine.start(SqlLine.java:267)
	at sqlline.SqlLine.main(SqlLine.java:206)
\end{lstlisting} \ \newline%
I am not sure what the problem is exactly. My guess is that Phoenix tries to optimize (flatten) an inner{-}query, which it shouldn't, if we are inside a join (according to the check in the code which throws the exception).\newline%
\newline%
The best workaround I found was to define an explicit join in the original query (Query 1), basically change the inner select into a join. This modified query return the same as the original one:\newline%
\newline%
\newline%
*Query 3:*\newline%
\begin{lstlisting}[language=sql]
SELECT T.id, T.test_id, T.lastchangedFROM  test T  LEFT JOIN (
    SELECT max(lastchanged) AS max_timestamp,           test_id AS max_timestamp_test_id
    FROM test
    GROUP BY test_id
  ) JOIN_TABLE ON JOIN_TABLE.max_timestamp_test_id = T.test_id
WHERE T.lastchanged = JOIN_TABLE.max_timestamp

Returns:
+------+-----------+-----------------------+
| T.ID | T.TEST
_ID |     T.LASTCHANGED     |
+------+-----------+-----------------------+
| 1    | 101       | 2000-01-01 01:00:00.0 |
| 2    | 100       | 2011-11-11 12:11:11.0 |
+------+-----------+-----------------------+
\end{lstlisting} \ \newline%
*Query 4:* And the same modified query (query 3) now works inside a join:\newline%
\begin{lstlisting}[language=sql]
SELECT AAA.*
FROM(
  SELECT T.id, T.test_id, T.lastchanged  FROM    test T    LEFT JOIN (
      SELECT max(lastchanged) AS max_timestamp,             test_id AS max_timestamp_test_id
      FROM test
      GROUP BY test_id
    ) JOIN_TABLE ON JOIN_TABLE.max_timestamp_test_id = T.test_id
  WHERE T.lastchanged = JOIN_TABLE.max_timestamp
) as AAA,
(
  SELECT id FROM te
st
) as BBB
WHERE AAA.id = BBB.id;

Returns:
+------+-----------+-----------------------+
| T.ID | T.TEST_ID |     T.LASTCHANGED     |
+------+-----------+-----------------------+
| 1    | 101       | 2000-01-01 01:00:00.0 |
| 2    | 100       | 2011-11-11 12:11:11.0 |
+------+-----------+-----------------------+
\end{lstlisting} \ \newline%
\newline%
I think Query 4 worked, as it is forcing Phoenix to drop the idea of optimizing it's inner{-}query (Query 3). Although, I can be wrong about the root cause...\newline%
\newline%
Anyway, I think the bug should be fixed and Query 2 should run without exception.

%
\section{Attachments}%
\label{sec:Attachments}%
\begin{enumerate}%
\item%
\href{https://issues.apache.org/jira/secure/attachment/13016278/PHOENIX-6232_v1-4.x.patch}{\underline{PHOENIX{-}6232\_v1{-}4.x.patch}}%
\item%
\href{https://issues.apache.org/jira/secure/attachment/13016332/PHOENIX-6232_v1-master.patch}{\underline{PHOENIX{-}6232\_v1{-}master.patch}}%
\end{enumerate}

%
\section{Comments}%
\label{sec:Comments}%
\begin{enumerate}%
\item%
\textbf{comnetwork: }{[}\textasciitilde{}symat{]}, thank you very much for the report.  It  is indeed a bug, I would try to fix it.\newline%
\newline%
I reproduced your problem on the latest branch 4.x, but the exception is different from your reported exception, what is your phoenix version? %
\item%
\textbf{symat: }{[}\textasciitilde{}comnetwork{]}, thanks for looking into this!\newline%
\newline%
\newline%
\newline%
>I reproduced your problem on the latest branch 4.x, but the exception is different from your reported exception, what is your phoenix version?\newline%
\newline%
\newline%
\newline%
First I saw this issue on our downstream phoenix, which is based on 5.0.0 (but might contain some newer commits cherry{-}picked). Then I tried to reproduce the same problem on other versions. The stacktrace I copied here come froma phoenix I built from the apache master branch 2 days ago. (I haven't tested this on any 4.x versions yet)%
\item%
\textbf{comnetwork: }{[}\textasciitilde{}symat{]}. thank you very much, I would try to fix this and to test   on the 4.x and master.%
\item%
\textbf{wangchao316: }I reproduct this issues in 5.0.0 branch. this exception is same as query 2.%
\item%
\textbf{githubbot: }comnetwork opened a new pull request \#992:\newline%
URL: https://github.com/apache/phoenix/pull/992\newline%
\newline%
\newline%
  \newline%
\newline%
{-}{-}{-}{-}{-}{-}{-}{-}{-}{-}{-}{-}{-}{-}{-}{-}{-}{-}{-}{-}{-}{-}{-}{-}{-}{-}{-}{-}{-}{-}{-}{-}{-}{-}{-}{-}{-}{-}{-}{-}{-}{-}{-}{-}{-}{-}{-}{-}{-}{-}{-}{-}{-}{-}{-}{-}{-}{-}{-}{-}{-}{-}{-}{-}\newline%
This is an automated message from the Apache Git Service.\newline%
To respond to the message, please log on to GitHub and use the\newline%
URL above to go to the specific comment.\newline%
\newline%
For queries about this service, please contact Infrastructure at:\newline%
users@infra.apache.org\newline%
%
\item%
\textbf{comnetwork: }{[}\textasciitilde{}symat{]}, I uploaded my patch to fix this bug, I think the root cause is that\newline%
\begin{lstlisting}[language=java]

 SELECT id, test_id, lastchanged FROM test T

  WHERE lastchanged = ( SELECT max(lastchanged) FROM test WHERE test_id = T.test_id )

\end{lstlisting} \ \newline%
\newline%
\newline%
\newline%
has  Correlated subquery (\{\{SELECT max(lastchanged) FROM test WHERE test\_id = T.test\_id\}\}),  which is a join itself, so it could not as the probe side of the Hash join.\newline%
\newline%
\newline%
\newline%
You may also use /*+ USE\_SORT\_MERGE\_JOIN*/ hint to get around this bug.%
\item%
\textbf{symat: }{[}\textasciitilde{}comnetwork{]}, thanks for the very quick fix! :)\newline%
\newline%
\newline%
\newline%
I'm going to build your patch and also execute the original query failed for us in production.%
\item%
\textbf{githubbot: }symat commented on pull request \#992:\newline%
URL: https://github.com/apache/phoenix/pull/992\#issuecomment{-}736616782\newline%
\newline%
\newline%
   I can not comment on the code (I'm not familiar with Phoenix internals) but I executed the original production query that failed for us before the patch, and after applying this patch the query succeed. Thanks for the fix!\newline%
\newline%
\newline%
{-}{-}{-}{-}{-}{-}{-}{-}{-}{-}{-}{-}{-}{-}{-}{-}{-}{-}{-}{-}{-}{-}{-}{-}{-}{-}{-}{-}{-}{-}{-}{-}{-}{-}{-}{-}{-}{-}{-}{-}{-}{-}{-}{-}{-}{-}{-}{-}{-}{-}{-}{-}{-}{-}{-}{-}{-}{-}{-}{-}{-}{-}{-}{-}\newline%
This is an automated message from the Apache Git Service.\newline%
To respond to the message, please log on to GitHub and use the\newline%
URL above to go to the specific comment.\newline%
\newline%
For queries about this service, please contact Infrastructure at:\newline%
users@infra.apache.org\newline%
%
\item%
\textbf{symat: }I executed the original production query that failed for us before the patch, and after applying this patch the query succeed. I tested it on branch 4.x, as the PR was submitted against 4.x. Thanks for the fix again!\newline%
\newline%
\newline%
\newline%
Do you plan to also cherry{-}pick the change to master? (I don't know how the branching works in Phoenix... but it would be great to have this fix in the next 5.x release too)%
\item%
\textbf{githubbot: }stoty commented on pull request \#992:\newline%
URL: https://github.com/apache/phoenix/pull/992\#issuecomment{-}736651838\newline%
\newline%
\newline%
   :broken\_heart: **{-}1 overall**\newline%
\newline%
  \newline%
  \newline%
  \newline%
  \newline%
  \newline%
  \newline%
   | Vote | Subsystem | Runtime | Comment |\newline%
\newline%
   |:{-}{-}{-}{-}:|{-}{-}{-}{-}{-}{-}{-}{-}{-}{-}:|{-}{-}{-}{-}{-}{-}{-}{-}:|:{-}{-}{-}{-}{-}{-}{-}{-}|\newline%
\newline%
   | +0 :ok: |  reexec  |   5m  7s |  Docker mode activated.  |\newline%
\newline%
   ||| \_ Prechecks \_ |\newline%
\newline%
   | +1 :green\_heart: |  dupname  |   0m  0s |  No case conflicting files found.  |\newline%
\newline%
   | +1 :green\_heart: |  hbaseanti  |   0m  0s |  Patch does not have any anti{-}patterns.  |\newline%
\newline%
   | +1 :green\_heart: |  @author  |   0m  0s |  The patch does not contain any @author tags.  |\newline%
\newline%
   | +1 :green\_heart: |  test4tests  |   0m  0s |  The patch appears to include 2 new or modified test files.  |\newline%
\newline%
   ||| \_ 4.x Compile Tests \_ |\newline%
\newline%
   | +1 :green\_heart: |  mvninstall  |  11m 16s |  4.x passed  |\newline%
\newline%
   | +1 :green\_heart: |  compile  |   0m 55s |  4.x passed  |\newline%
\newline%
   | +1 :green\_heart: |  checkstyle  |   1m 33s |  4.x passed  |\newline%
\newline%
   | +1 :green\_heart: |  javadoc  |   0m 43s |  4.x passed  |\newline%
\newline%
   | +0 :ok: |  spotbugs  |   2m 56s |  phoenix{-}core in 4.x has 950 extant spotbugs warnings.  |\newline%
\newline%
   ||| \_ Patch Compile Tests \_ |\newline%
\newline%
   | +1 :green\_heart: |  mvninstall  |   5m 24s |  the patch passed  |\newline%
\newline%
   | +1 :green\_heart: |  compile  |   0m 55s |  the patch passed  |\newline%
\newline%
   | +1 :green\_heart: |  javac  |   0m 55s |  the patch passed  |\newline%
\newline%
   | {-}1 :x: |  checkstyle  |   1m 39s |  phoenix{-}core: The patch generated 121 new + 2612 unchanged {-} 83 fixed = 2733 total (was 2695)  |\newline%
\newline%
   | +1 :green\_heart: |  whitespace  |   0m  0s |  The patch has no whitespace issues.  |\newline%
\newline%
   | {-}1 :x: |  javadoc  |   0m 41s |  phoenix{-}core generated 1 new + 99 unchanged {-} 1 fixed = 100 total (was 100)  |\newline%
\newline%
   | +1 :green\_heart: |  spotbugs  |   3m  6s |  the patch passed  |\newline%
\newline%
   ||| \_ Other Tests \_ |\newline%
\newline%
   | {-}1 :x: |  unit  | 206m 57s |  phoenix{-}core in the patch failed.  |\newline%
\newline%
   | +1 :green\_heart: |  asflicense  |   0m 35s |  The patch does not generate ASF License warnings.  |\newline%
\newline%
   |  |   | 244m 42s |   |\newline%
\newline%
  \newline%
  \newline%
   | Reason | Tests |\newline%
\newline%
   |{-}{-}{-}{-}{-}{-}{-}:|:{-}{-}{-}{-}{-}{-}|\newline%
\newline%
   | Failed junit tests | phoenix.end2end.PointInTimeQueryIT |\newline%
\newline%
  \newline%
  \newline%
   | Subsystem | Report/Notes |\newline%
\newline%
   |{-}{-}{-}{-}{-}{-}{-}{-}{-}{-}:|:{-}{-}{-}{-}{-}{-}{-}{-}{-}{-}{-}{-}{-}|\newline%
\newline%
   | Docker | ClientAPI=1.40 ServerAPI=1.40 base: https://ci{-}hadoop.apache.org/job/Phoenix/job/Phoenix{-}PreCommit{-}GitHub{-}PR/job/PR{-}992/1/artifact/yetus{-}general{-}check/output/Dockerfile |\newline%
\newline%
   | GITHUB PR | https://github.com/apache/phoenix/pull/992 |\newline%
\newline%
   | JIRA Issue | PHOENIX{-}6232 |\newline%
\newline%
   | Optional Tests | dupname asflicense javac javadoc unit spotbugs hbaseanti checkstyle compile |\newline%
\newline%
   | uname | Linux a25f0bde3789 4.15.0{-}65{-}generic \#74{-}Ubuntu SMP Tue Sep 17 17:06:04 UTC 2019 x86\_64 x86\_64 x86\_64 GNU/Linux |\newline%
\newline%
   | Build tool | maven |\newline%
\newline%
   | Personality | dev/phoenix{-}personality.sh |\newline%
\newline%
   | git revision | 4.x / 18b9f76 |\newline%
\newline%
   | Default Java | Private Build{-}1.8.0\_242{-}8u242{-}b08{-}0ubuntu3\textasciitilde{}16.04{-}b08 |\newline%
\newline%
   | checkstyle | https://ci{-}hadoop.apache.org/job/Phoenix/job/Phoenix{-}PreCommit{-}GitHub{-}PR/job/PR{-}992/1/artifact/yetus{-}general{-}check/output/diff{-}checkstyle{-}phoenix{-}core.txt |\newline%
\newline%
   | javadoc | https://ci{-}hadoop.apache.org/job/Phoenix/job/Phoenix{-}PreCommit{-}GitHub{-}PR/job/PR{-}992/1/artifact/yetus{-}general{-}check/output/diff{-}javadoc{-}javadoc{-}phoenix{-}core.txt |\newline%
\newline%
   | unit | https://ci{-}hadoop.apache.org/job/Phoenix/job/Phoenix{-}PreCommit{-}GitHub{-}PR/job/PR{-}992/1/artifact/yetus{-}general{-}check/output/patch{-}unit{-}phoenix{-}core.txt |\newline%
\newline%
   |  Test Results | https://ci{-}hadoop.apache.org/job/Phoenix/job/Phoenix{-}PreCommit{-}GitHub{-}PR/job/PR{-}992/1/testReport/ |\newline%
\newline%
   | Max. process+thread count | 6170 (vs. ulimit of 30000) |\newline%
\newline%
   | modules | C: phoenix{-}core U: phoenix{-}core |\newline%
\newline%
   | Console output | https://ci{-}hadoop.apache.org/job/Phoenix/job/Phoenix{-}PreCommit{-}GitHub{-}PR/job/PR{-}992/1/console |\newline%
\newline%
   | versions | git=2.7.4 maven=3.3.9 spotbugs=4.1.3 |\newline%
\newline%
   | Powered by | Apache Yetus 0.12.0 https://yetus.apache.org |\newline%
\newline%
  \newline%
  \newline%
   This message was automatically generated.\newline%
\newline%
  \newline%
  \newline%
\newline%
{-}{-}{-}{-}{-}{-}{-}{-}{-}{-}{-}{-}{-}{-}{-}{-}{-}{-}{-}{-}{-}{-}{-}{-}{-}{-}{-}{-}{-}{-}{-}{-}{-}{-}{-}{-}{-}{-}{-}{-}{-}{-}{-}{-}{-}{-}{-}{-}{-}{-}{-}{-}{-}{-}{-}{-}{-}{-}{-}{-}{-}{-}{-}{-}\newline%
This is an automated message from the Apache Git Service.\newline%
To respond to the message, please log on to GitHub and use the\newline%
URL above to go to the specific comment.\newline%
\newline%
For queries about this service, please contact Infrastructure at:\newline%
users@infra.apache.org\newline%
%
\item%
\textbf{comnetwork: }{[}\textasciitilde{}symat{]},Thank you for quick feedback, I would make a patch for the master also.%
\item%
\textbf{comnetwork: }Uploaded my patch for the master.%
\item%
\textbf{symat: }Thank you {[}\textasciitilde{}comnetwork{]}!! :)%
\end{enumerate}

%
\end{document}